% main.tex - 主文档
% 在Xelatex下编辑
\documentclass[10pt, twoside]{article}
\usepackage[UTF8]{ctex} % 支持中英文
\usepackage[top=2cm, bottom=2.5cm, left=1.5cm, right=1.5cm]{geometry}
\usepackage{graphicx}
\usepackage{tabularx}
\usepackage{booktabs}
\usepackage{float}
\usepackage{caption}
\usepackage{fancyhdr}
\usepackage{setspace}
\usepackage{amsmath}
\usepackage{indentfirst}
\usepackage{enumitem}
\usepackage{amssymb}
\setlength{\parindent}{1.25em}

% 设置中文字体(避免Overleaf免费版字体缺失问题)
\setCJKmainfont{Noto Serif CJK SC} % 可使用Overleaf预装的开源字体
\setCJKsansfont{Noto Sans CJK SC}
\setCJKmonofont{Noto Sans Mono CJK SC}

% 目录超链接
\usepackage{hyperref}
\hypersetup{
    colorlinks=true, % false: 显示边框,true: 显示彩色文字
    linkcolor=blue,
    filecolor=magenta,      
    urlcolor=cyan,
    pdftitle={Research Pamphlet},
    bookmarks=true,
}

% 设置页眉页脚
\pagestyle{fancy}
\fancyhead[L]{Xiao Dongfu}
\fancyhead[C]{City University of Macau}
\fancyhead[R]{Research Pamphlet}

% 封面页定义
\newcommand{\makecover}{
    \thispagestyle{empty}
    \begin{center}
        \vspace{0.3cm}
        \includegraphics[width=1\textwidth]{cityu.png}
        \vspace{1.1cm}
        
        {\fontsize{22}{25}\selectfont\bfseries ENGE3400A $|$ 计算文学研究导论}
        
        \vspace{2cm}
       
        {\fontsize{20}{19}\selectfont{\textbf{研究手册}}}\\[0.9cm]
        
        \vspace{1.8cm}
        {\fontsize{22}{25}\selectfont\underline{{\textit{20世纪儿童文学中的情感模式研究}}}}
        
        % 作者信息
        \vspace{4cm}
        {
        \fontsize{16}{19}\selectfont
        \begin{tabular}{c}
            \hline \\[-1.2ex]
            \textbf{\Large 肖东福} \\ 
            \textbf{数据科学学院} \\
            \textbf{\Large 澳门城市大学} \\ 
            \hline 
        \end{tabular}
        }
        \vspace{1.5cm}
        
        {\large{\today}}
    \end{center}
    \clearpage
}

\begin{document}

% 封面页
\makecover

% 目录
\tableofcontents
\newpage

% 正文内容
\section{引言} 
本研究旨在探讨......
\subsection{xxx} % 字标题
% 表格例子
\begin{table}[H] % 【H】强制它在你写这个代码的当前位置显示图片或表格 如果用使用[htbp]:让LaTeX决定最佳位置,所以图片或表格会跑到其他文字的位置
\caption{Corpus composition by period}
\label{tab:corpus}
\centering % 图片剧中
\begin{tabular}{lrr}
\toprule
\textbf{Period} & \textbf{Texts} & \textbf{Words (millions)} \\
\midrule
Renaissance & 127 & 12.4 \\
Victorian & 94 & 24.7 \\
Modernist & 63 & 8.2 \\
\bottomrule
\end{tabular}
\end{table}
% 看到end表示表格结束
\section{研究方法}
词频与语义重要性之间的关系可表示为:
\[
f(w) = \frac{n_w}{N} \times \log\left(\frac{N}{d_w}\right)
\]
其中:
\begin{itemize}
    \item $f(w)$ 是词$w$的重要性分数
    \item $n_w$ 是$w$的出现次数
    \item $N$ 是语料库中的总词数
    \item $d_w$ 是包含$w$的文档数量
\end{itemize}


\section{结果}
% 结果内容...

\section{限制}
% 限制内容...

\section{结论}
这是一个引用示例\cite{moretti}

% 参考文献
\begin{thebibliography}{9}
\bibitem{moretti} 
Moretti, F. (2013). \textit{Distant Reading}. Verso Books.
\end{thebibliography}
\end{document}